%%%%%%%%%%%%%%%%%%%%%%%%%%%%%%%%%%%%%%%%%
% This template requires the resume.cls file to be in the same directory as the
% .tex file. The resume.cls file provides the resume style used for structuring the
% document.
%%%%%%%%%%%%%%%%%%%%%%%%%%%%%%%%%%%%%%%%%
\documentclass{resume} % Use the custom resume.cls style

\usepackage[left=0.5in,top=0.3in,right=0.5in,bottom=0.3in]{geometry} % Document margins
\usepackage{bibentry}

\name{Joshua Palamuttam} % Your name
\address{San Jose, CA 95138} % Your address
\address{(408)~$\cdot$~841~$\cdot$~6194 \\ joshpal97@gmail.com}%Your phone number and email
\address{https://github.com/joshpal97 \\ https://JoshuaPalamuttam.com}

\begin{document}
%----------------------------------------------------------------------------------------
%	SUMMARY SECTION
%----------------------------------------------------------------------------------------
\begin{rSection}{Summary}
{A highly-motivated computer professional, who graduated with a Computer Science degree from Rose-Hulman Institute of Technology '19, 
with the enthusiasm to accept new challenges. The software engineering positions I have been in have given me hands-on work experience 
working on application design, development, and quality assurance using agile practices, and writing clean code.}
\end{rSection}

%----------------------------------------------------------------------------------------
%	EDUCATION SECTION
%----------------------------------------------------------------------------------------
\begin{rSection}{Education}
\begin{rsemisection}{Rose-Hulman Institute of Technology}{2015 - 2019}{Bachelor of Science in Computer Science}
\end{rsemisection}
\end{rSection}
%----------------------------------------------------------------------------------------
%	WORK EXPERIENCE SECTION
%----------------------------------------------------------------------------------------

\begin{rSection}{Experience}
\begin{rSubsection}{Kaiser Permanente Genetics}{February 2020 - Present}{Junior Dot Net Analyst, San Jose, CA}

\item Main purpose was to do R&D for different applications used by the Genetics Department
\item Responsibilities for this job include gathering requirements along with developing and supporting applications
\item Made applications with the Kaiser tech stack of primarily .Net Core MVC applications, with a SQL Database
\item Occasionally made Batch Files, or Excel VBA scripts to aid Lab Technicians in thier duties
\end{rSubsection}
\begin{rSubsection}{DMI}{May 2019 - February 2020}{Software Engineer, Indianapolis, IN}

\item Helped create a POC for Kar Global to optimize lot operations at a car auction site, using Angular, Node.js, AWS serverless, SQL (2 months)
\item Helped build a physics engine called ISCAAN for Allison Transmission to simulate a transmission (7 months)
\item The majority of Allison Transmission's proprietary math was done using C++, which was then passed to a Angular, .Net Core, SQL stack, using Entity Framework Core 
\item Because of ISCAAN's large nature, the project was designed to use a hexagonal software architecture
\item Built Microsoft Teams Chat bot for DMI, using LUIS an Azure Cognitive Service (NLP) with a .Net Core back end (1 month)
\item All projects used Azure Devops to adhere to SAFe agile practices, such as CI/CD, unit and integration testing
\end{rSubsection}

\end{rSection}

%----------------------------------------------------------------------------------------
%	PROJECTS SECTION
%----------------------------------------------------------------------------------------

\begin{rSection}{Projects}
\begin{rsemisection}{Compile.io (Angular, Spring Boot, MongoDb, Docker)}{2018-2019}
\item My Senior Project, tasked to build a scalable website that automatically grades student submitted code
\item Code was submitted through the Angular front-end, taken to our Java Spring backend, processed in a docker container, and stored in MongoDB.
\item Gathered requirements from professors, and wrote documentation that can be viewed in my github
\item Followed Agile methodologies, with CI/CD, 2 week sprints, feature branching strategy, and unit testing
\end{rsemisection}

\begin{rsemisection}{Stock Market Analysis through Twitter (Python) }{Spring 2019}
    \item The purpose of this project was to figure out how a company’s twitter handle and company’s CEO twitter handle affected the company’s stock
    \item Performed Latent Semantic Analysis on tweets for those twitter handles
    \item Pre-processed data by removing stop words, duplicate tweets, and special characters
    \item Used cross-validation, and Naive Bayes to predict the Stock Market
    \item The stock data was recieved through quandl, and the tweets were recieved through tweepy
\end{rsemisection}

\begin{rsemisection}{Full stack website, College Cookbook (Javascript, Firebase, MongoDB)}{Fall 2017}
\item The website can be found at this link: https://college-cookbook-jkl.firebaseapp.com/login.html or github
\item Developed and Integrated a MongoDB database for the back end and deployed front end through Firebase
\item Features: posting to twitter, emailing recipes, inputting recipes into the database, and secure login
\end{rsemisection}
\end{rSection}

%----------------------------------------------------------------------------------------
%	TECHNICAL STRENGTHS SECTION
%----------------------------------------------------------------------------------------
\begin{rSection}{Skills}
\begin{tabular}{ @{} >{\bfseries}l @{\hspace{7ex}} l }
Programming Languages & JavaScript, Typescript, C\#, Java, SQL, HTML, CSS, Python\\[0.001ex]
                      %& R \\[0.001ex]
Frameworks & Angular, Node.js, Spring, .Net Core, Entity Framework Core \\[0.001ex]
Tools & AWS, Azure Devops, Docker, Git, Visual Studio, Visual Studio Code \\[0.001ex]
Certifications & AWS Cloud Practitioner, SAFe 4.6
\end{tabular}
\end{rSection}

\end{document}
